\chapter{Methods}
\label{chap:methods}

\hspace{0pt}
\vfill
\section{Animals and Animal Preparation}
Experiments were performed using wood ants (\textit{Formica rufa L.}). Colonies were collected during summer 2018 from Ashdown Forest, Sussex, UK (N 51 4.680, E 0 1.800). They were kept in large plastic containers with Fluon coated walls at 26 degrees celsius with a 12h light: 12h darkness light cycle regime. They were fed \textit{ad libitum} with 33.3\% sucrose solution. \\
For experiments, ants were chosen based on their size and level of activity (large, active ants were preferred). Subsequently, these ants were painted and placed in small groups for 30 minutes to reduce ant-ant aggression due to the paint odour before being transferred to a small portable nest-like box with unpainted nest mates. To prepare an ant for the experiment, they were first put on ice for 5-10 min to cool them reducing movement before an insect pin (Austerlitz Insect Pin) was attached to the ants back with ultraviolet-light-sensitive glue (5 Second Fix) under a microscope (Olympus Corporation, Tokyo, JP). The pin was bent to a 135\textsuperscript{o} angle about 2/3 up the pin. The ant was then transferred to a separate box where it was kept until the first trial. After the first trial the ant was transferred to nest-like box and kept for 2-4h before the second trial. Ants were picked up with tweezers by the pin and harnessed to the trackball system (described in detail later). After finishing the experiments, animals were discarded in ethanol ($\geq$ 99.8\%, Sigma-Aldrich Ltd, Dorset, UK).
\vfill
\hspace{0pt}