\chapter{Background}
\label{chap:intro}
\vspace{4cm}
Locomotion is the means by which animals move around the world. However, movement comes at a price. Sensory performance is decreased as motion blur hampers visual discrimination \autocites{Kramer2001, Land1999} and extraction of relevant visual information may prove more difficult \autocite{Land1999}. This is particularly important in unpredictable environments. Furthermore, it is dependent upon muscular work and it is thus inherently energetically costly. An example is the 15-50 fold increase in oxygen consumption in flying locusts compared to rest-state \autocite{Krogh1949TheFlight}.
It is thus reasonable to believe that metrics of walking are evolutionarily optimised. That would entail animals choosing walking speeds that allows extraction of visual information, such as distance cues, whilst also being energetically efficient. \\
\vspace{1cm}



In this chapter, I will review the:
\begin{itemize}[noitemsep]
    \item Evidence for visual control of locomotion
    \item Evidence for energetic control of locomotion
    \item The use of virtual reality in animal behaviour research
\end{itemize}

\vfill
\clearpage
\section{Control of locomotion}

\subsection{Optic flow: Definition and features}

Optic flow is the movement of structured light across the retina \autocite{Raudies2013}. The practical meaning of this is that optic flow does not estimate the physical speed of things but rather the speed at which they move past the eye (in degrees/s). Optic flow can be divided roughly into two parts, rotational and translational both of which can occur around all axes. Both contain information about the speed and direction of movement. Translational optic flow from forward movement is generated around a central focal point with vectors pointing in opposite directions for the two eyes. During forward motion, the flow points away from the focal point and is said to be expansive. During backwards motion, on the other hand, it points towards the focal point and is retracting. Translational optic flow provides information regarding the translational distance of movement, and also allows for inference of depth due to the relative movement of components across the visual field, called motion parallax (e.g. when an animal moves forward, the translational optic flow generated by objects nearby will be larger than that generated by objects far away). Insects like locusts and mantids use motion parallax by translating sideways (peering) without rotating, to infer distance (e.g. \cite{Sobel1990b}). In rotational optic flow, vectors are pointing in the same direction for both eyes, providing directional information. However, it does not allow for depth perception as retinal displacements are independent of the distance to the retina. As movement is often a mixture of simultaneous translation and rotation rather than pure translation, this makes efficient extraction of distance information more difficult. In this thesis, I only discuss translational optic flow unless otherwise stated.\\ \\
Optic flow was first proposed in a biological setting by Gibson in 1958 \autocite{Gibson1958a} and has since been confirmed to be behaviourally relevant in a wealth of species, including humans \autocite{Warren1988}, rats \autocite{Kautzky2016}, zebra fish \autocite{Wang2019}, crabs \autocite{Horseman2011}, bees \autocite{Linander2015}, blowflies \autocite{Longden2009} and desert ants \autocite{Pfeffer2016}. Furthermore, it has been shown to influence a multitude of behaviours such as collision avoidance, mate following, predator avoidance, feature detection and path integration (e.g. \cite{Srinivasan1996}).

Within insects, \textcite{David1982CompensationTunnel} was the first to investigate the use of optic flow in the control of locomotion speed in a series of experiments on \textit{Drosophila}. It had previously been shown that some flying insects can maintain their flight speeds constant relative to the ground \autocite{David1979OptomotorDrosophila.}. \textcite{David1982CompensationTunnel} made flies fly inside a "barber's pole" wind tunnel, enabling him to manipulate the visual motions around the flies. He was able to make the flies adopt a stable position within the tunnel by imposing a headwind in the tunnel. Flies kept a steady ground speed in head winds ranging from 0.2-1.0m/s. He then manipulated the diameter of the tunnel, the wavelength of the revolving pattern and the pitch of the revolving lines. Thus, he could infer which aspects enabled to fly to maintain a preferred speed and concluded that it was due to the optic flow.

These findings were extended to another flying species, the honey bee, by \textcite{Srinivasan1996}. They had previously shown that honey bees use optic flow in their centering response, which enables them to keep objects on either side at equal distances \autocite{Srinivasan1991RangeHoneybees}. This response is a way of keeping them from colliding with nearby objects. They now set out to investigate if optic flow was also responsible of the control of their flight speed. By showing that bees slow down when flying through a narrowing gap with vertical lines on the walls, they showed that bees do indeed use optic flow. To control that it was not confounded by the fact that closer lines appear larger, they had bees fly through a tunnel in which the lines had a short wavelength and the other had a greater wavelength. Bees showed no change of flight speed, confirming that optic flow was indeed the metric used by the bees. The findings have since also been confirmed in bumblebees \autocite{Linander2015}. 